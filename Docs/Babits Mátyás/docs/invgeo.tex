\documentclass{article}
\usepackage[utf8]{inputenc}
\usepackage{amsmath}
\usepackage{graphicx}
\graphicspath{ {images/} }

\begin{document}

\section{Inverz geometria}
$A sin(\vartheta)+B cos(\vartheta)=D$ egyenlet megoldásai:
\begin{align} \label{eq:trigeq}
\vartheta = arctan \left( \frac{A D\pm B\sqrt{A^2+B^2-D^2}}{B D\mp A\sqrt{A^2+B^2-D^2}} \right)
\end{align}
Az $arctan$ függvény helyett használható az $atan2()$ függvény, ami külön kapja meg a számlálót és nevezőt, és $2\pi$ periodicitással adja meg a szöget.
\subsection{Első szög}
Direkt geometriából ismertek a következő egyenletek:
\begin{align*}
C_1 d_{3x}+S_1 (C_{23} d_3+C_2 d_2+d_1)+d_x=p_x
\end{align*}
\begin{align*}
-S_1 d_{3x}+C_1 (C_{23} d_3+C_2 d_2+d_1)+d_x=p_z
\end{align*}
Ezekből:
\begin{align*}
\frac{C_1}{S_1}d_{3x}+(C_{23} d_3+C_2 d_2+d_1)=\frac{p_x-d_x}{S_1}
\end{align*}
\begin{align*}
\frac{-S_1}{C_1}d_{3x}+(C_{23} d_3+C_2 d_2+d_1)=\frac{p_z-d_z}{C_1}
\end{align*}
A két egyenletet kivonva egymásból és megszorozva $C_1 S_1$-gyel:
\begin{align*}
(S_1^2+C_1^2 ) d_3x=C_1 (p_x-d_x )-S_1 (p_z-d_z)
\end{align*}
Itt a \eqref{eq:trigeq} egyenletbe behelyettesítve $A=d_z-p_z$, $B=p_x-d_x$ és $D=d_3x$ értékeket kapható az első csukló állása.
\begin{align}
\vartheta_1 = arctan \left( \frac{(d_z-p_z) d_3\pm (p_x-d_x)\sqrt{(d_z-p_z)^2+(p_x-d_x)^2-d_3^2}}{(p_x-d_x) d_3\mp (d_z-p_z)\sqrt{(d_z-p_z)^2+(p_x-d_x)^2-d_3^2}} \right)
\end{align}
 Mivel $D\cong0 (4 mm)$, $\vartheta_1\cong arctan\left(\frac{\pm B}{\mp A}\right)$, így a két megoldás közel $\pi$-ben tér el egymástól. Emiatt, és mert a határok közelében a többi láb mozgásterét zavarná, és így azt nem éri, elég a $-\pi/2$, $\pi/2$ közötti eredménnyel folytatni a számolást.
\subsection{Második szög}
\begin{align*}
C_{23} d_3+C_2 d_2=\frac{p_x-d_x-C_1 d_{3x}}{S_1} -d_1=l
\end{align*}
\begin{align*}
S_{23} d_3+S_2 d_2=d_y-p_y
\end{align*}
$C_{23}$-at és $S_{23}$-at kifejezve:
\begin{align}\label{eq:c23eq}
C_{23}=\frac{l-C_2 d_2}{d_3}
\end{align}
\begin{align}\label{eq:s23eq}
S_{23}=\frac{d_y-p_y-S_2 d_2}{d_3} 
\end{align}
Ezeket négyzetre emelve:
\begin{align*}
C_{23}^2=\frac{l^2+C_2^2 d_2^2-2lC_2 d_2}{d_3^2}
\end{align*}
\begin{align*}
S_{23}^2=\frac{\left(d_y-p_y\right)^2+S_2^2d_2^2-2\left(d_y-p_y\right)S_2d_2}{d_3^2}
\end{align*}
Összeadva a két egyenletet
\begin{align*}
1=\frac{l^2+(C_2^2+S_2^2)d_2^2+\left(d_y-p_y\right)^2-2lC_2 d_2-2\left(d_y-p_y\right)S_2 d_2)}{d_3^2}
\end{align*}
\begin{align*}
e=\frac{l^2+d_2^2+\left(d_y-p_y\right)^2-d_3^2}{2d_2}=lC_2+\left(d_y-p_y\right)S_2
\end{align*}
$A=d_y-p_y$ $B=l$ $D=e$ helyettesítésekkel a képletbe behelyettesítéssel két eredmény kapható.
\begin{align}
\vartheta_2 = arctan \left( \frac{(d_y-p_y) e\pm l\sqrt{(d_y-p_y)^2+l^2-e^2}}{l e\mp (d_y-p_y)\sqrt{(d_y-p_y)^2+l^2-e^2}} \right)
\end{align}
Ezek közül az egyiknél a robot térde felfelé hajlik, a másik megoldásban lefelé. Mindkét megoldással tovább lehet haladni.
\subsection{Harmadik szög}
\eqref{eq:c23eq} és \eqref{eq:s23eq} képleteket elosztva egymással, kifejezve $\vartheta_3$-ra
\begin{align}
\vartheta_3=arctan\left(\frac{d_y-p_y-S_2 d_2}{l-C_2 d_2}\right)
\end{align}
\subsection{Eredmények értékelése}
Ezáltal kaptunk két megoldást a kívánt pozícióra. Az offseteket kivonva ellenőrizhető, hogy a motor tartományán belül van-e az eredmény. Ha nincs, érvénytelen. Ha a láb hatótávolságán kívül adtunk meg pozíciót, a másodfokú egyenlet diszkriminánsa negatív lesz, nem hozva eredményt.



\end{document}