\documentclass{article}
\usepackage[utf8]{inputenc}
\usepackage{amsmath}
\usepackage{graphicx}
\graphicspath{ {images/} }

\begin{document}
	
Mozgás\\\\
A quadruped mozgásáért két osztály (WalkScript, WalkManager) és egy struktúra (QuadAction) felelős. A WalkScript tárolja a lábak mozgatásával kapcsolatos információkat.\\

WalkScript\\
\begin{tabular}{| l | l | l |}
\hline
változó&típus&leírás\\
\hline
maxTurnAtOnce&float&legnagyobb engedett fordulás egyszerre radiánban\\
\hline
bellyy&float&robot aljának távolsága a talajtól\\
\hline
legLift&float&láb emelése lépés közben\\
\hline
legXPos&float&láb helyzete a robot oldalán, milyen szélesen lépjen\\
\hline
legZRetracted&float&lábat mennyire húzza be\\
\hline
legStretchHalf&float&lábat legtöbb mennyire nyújtsa ki osztva 2-vel\\
\hline
rightBalanced&bool&jobb oldalon vannak-e nagyobb egyensúlyban a lábak\\
\hline
script&fifo$<$QuadAction$>$&lépések sorozata fifoban tárolva\\
\hline
\end{tabular}\\
\\

WalkManager\\
\begin{tabular}{| l | l | l |}
	\hline
	változó&típus&leírás\\
	\hline
	time&float&adott lépésben helyzet [0..1]\\
	\hline
	speed&float&mozgás sebessége\\
	\hline
	script&WalkScript&lépéssorozat\\
	\hline
	quad&Quadruped&a quadruped amit mozgat\\
	\hline
	action&QuadAction&az elérendő helyzet\\
	\hline
	prevAction&QuadAction&kiinduló helyzet\\
	\hline
	running&bool&éppen mozog-e a robot\\
	\hline
\end{tabular}\\
\\

QuadAction\\
\begin{tabular}{| l | l | l |}
	\hline
	változó&típus&leírás\\
	\hline
	legID&int&lábazonosító, amelyikkel lép\\
	\hline
	goalPos&float2&elérendő pozíció\\
	\hline
	rot&float&elvégzendő forgás\\
	\hline
\end{tabular}\\\\

Mozgás előre

	
\end{document}